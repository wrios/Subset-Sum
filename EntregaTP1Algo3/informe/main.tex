\documentclass[11pt, a4paper]{article}
\usepackage[paper=a4paper, left=1.5cm, right=1.5cm, bottom=1.5cm, top=1.5cm]{geometry}
\usepackage[utf8]{inputenc}
\usepackage{clrscode3e}
\usepackage[spanish]{babel}
\newcommand\Frac[3][]{ #1 #2\color{black}\above 0.4pt \normalcolor #1#3}
\usepackage{amsmath,amssymb}
\usepackage{xcolor}
\usepackage{caratula/caratula}
\usepackage{bm}
\usepackage{wrapfig}
\usepackage{listings}
\usepackage{color}
 \def\infinity{\rotatebox{90}{8}}
\definecolor{codegreen}{rgb}{0,0.6,0}
\definecolor{codegray}{rgb}{0.5,0.5,0.5}
\definecolor{codepurple}{rgb}{0.58,0,0.82}
\definecolor{backcolour}{rgb}{0.95,0.95,0.92}
 
\lstdefinestyle{coding}{
    backgroundcolor=\color{backcolour},   
    commentstyle=\color{codegreen},
    keywordstyle=\color{magenta},
    numberstyle=\tiny\color{codegray},
    stringstyle=\color{codepurple},
    basicstyle=\footnotesize,
    breakatwhitespace=false,         
    breaklines=true,                 
    captionpos=b,                    
    keepspaces=true,                 
    numbers=left,                    
    numbersep=5pt,                  
    showspaces=false,                
    showstringspaces=false,
    showtabs=false,                  
    tabsize=1
}
 
\lstset{literate=%
{$}{{\$}}1
{å}{{\aa}}1
{ø}{{\o}}1
{Æ}{{\AE}}1
{Å}{{\AA}}1
{Ø}{{\O}}1
}
\lstset{extendedchars=\true}
\lstset{inputencoding=ansinew}
 
 
\lstset{style=coding}

\begin{document}

\titulo{Trabajo Práctico 1 }
\fecha{Domingo 16 de Septiembre de 2018}
\materia{Algoritmos y Estructuras de Datos III}

\integrante{Kennedy Williams Rios Cuba}{381/15}{wrios@dc.uba.ar}

%Carátula
\maketitle
\newpage

%Indice
\tableofcontents
\newpage

%Introduccion
\section{Introducción al Problema}

\subsection{Definicion del problema Subset Sum}
Dado un conjunto de n \'items $S$, cada uno con un $valor$ asociado $v_{i}$ , y un valor objetivo $V$ ,
decidir si existe un subconjunto de \'items de S que sumen exacto el valor objetivo $V$ . Si existe
dicho conjunto, decir cual es la m\'inima cardinalidad
P entre todos los subconjuntos posibles. \\
En otras palabras decidir si existe$ R \subseteq S$ tal que $\sum_{i \in R}v_{i} = V$, y si existe, devolver la menor cardinalidad posible de $R$.\\
Para este problema, asumiremos que todos los valores mencionados son enteros no negativos.
\subsection{Ejemplo de problema}

tomar el elemento(con un - arriba del nodo) para mirar si forma parte de la solución.
En este ejemplo se puede ver cual es el espacio de soluciones, la forma de recorrerlo será
definida más adelante  de acuerdo a cada algoritmo que usemos para resolver el problema.
Además se ver\'a que tamaño tiene este espacio, y la cantidad de posibles soluciones (en el diagrama trivialmente se logra ver como cada nodo hoja es una posible soluci\'on que llego desde las decisiones tomadas).
\subsubsection{Ejemplo 1}
En este ejemplo, se considera un conjunto de 4 elementos {11,7,5,2}.
Se busca que la suma de los elementos sea 18 y que el cardinal sea mínimo.
El nivel i representa la decisión de tomar el elemento(con un + arriba del nodo) o no
Como se puede ver en la imagen, hay dos soluciones {11,7} y {11,5,2} que suman 18 pero
la mejor es usando menos elementos, con lo cual nos quedamos con la solución {11,7}.
\begin{center}
\includegraphics[width=18cm, height=12cm]{diagrama.png}
\end{center}
\newpage
\subsubsection{Ejemplo 2}
Conjunto $\{2, 3, 12, 14, 4\}$, buscamos un subconjunto de elementos que sume 13.\\
El 14, no puede sumar 13. El 12 no se puede elegir porque no hay elementos que sumen 1. Al combinar cualquiera de los que restan, no alcanzan a sumar 13 con lo cual no hay soluci\'on para este problema.\\


\subsection{El objetivo del trabajo practico}
 Resolver el problema propuesto de diferentes maneras realizando posteriormente una comparaci\'on entre los diferentes algoritmos utilizados.
%\newpage

%Desarrollo
\section{Desarrollo Problema}
\subsection{Caracterizaci\'on de una soluci\'on}
Una solución de nuestro problema es un subconjunto de elementos del conjunto original tal que la suma de ellos es exactamente el valor objetivo.\\
\subsection{Espacio de soluciones}
Se redefine Solución para poder hablar de factibilidad y optimalidad.\\
En la imagen podemos ver, que luego de diferentes decisiones, llegamos a diferentes soluciones (hojas) y est\'an separados en tres casos.
\begin{itemize}
	\item Los nodos rojos, verdes y celestes son las soluciones.
	\item Los nodos verdes son las soluciones factibles.
	\item Los nodos rojos son las soluciones no factibles. 
\end{itemize}	
\begin{center}
\includegraphics[width=18cm, height=10cm]{3casos.png}
\end{center}
\subsection{Recorrido del espacio de soluciones}
Se ver\'an diferentes formas de pensar el problema, y en base a ello elegiremos una forma de obtener la soluci\'on.\\

\begin{itemize}
\item La primer forma de ver el problema es simple, mirar todo el espacio de soluciones y quedarnos con la mejor.\\
\item En la segunda forma trataremos de recorrer solamente el espacio de soluciones factibles.\\
\item La tercer forma va mas enfocado a cuando una soluci\'on es mejor que otra, y veremos como recorrer las soluciones que son mejores que la soluci\'on que tenemos hasta el momento (si las hay, sino podaremos).\\
\item La \'ultima forma de pensar el problema sera ver los problemas anteriores inmediatos y ver como con ellos se puede construir el problema mas grande.
\end{itemize}

%Fuerza Fruta
\section{Fuerza Bruta: Recorriendo todo el espacio}
\subsection{Algoritmo}


\begin{codebox}
    \Procname{\proc{FB}(vec<int> C, int V, int i, int n) \quad\quad\quad\quad\quad\textcolor{red}{O($2^n$)}}
    \li \If $i <= n$ \quad\quad\quad\quad\quad\quad\quad\quad\quad\quad\quad\quad\quad\textcolor{red}{O($1$)}
        \Then
            \Return $min(1+FB(C,V-C_{i},i+1,n),FB(C,V,i+1,n))$ \quad\quad\quad\quad\quad\quad\quad\quad
        \End
    \li \If $V = 0$ \quad\quad\quad\quad\quad\quad\quad\quad\quad\quad\quad\quad\quad\textcolor{red}{O($1$)}
        \Then
            \Return $0$ \quad\quad\quad\quad\quad\quad\quad\quad
        \End
    \li \If $i > n || V < 0 || v >0$ \quad\quad\quad\quad\quad\quad\quad\quad\quad\quad\quad\quad\quad\textcolor{red}{O($1$)}
        \Then
            \Return $min(1+FB(C,V-C_{i},i+1,n),FB(C,V,i+1,n))$ \quad\quad\quad\quad\quad\quad\quad\quad
        \End

    \end{codebox}
\subsection{Correctitud}
El algoritmo recorre todo el espacio de busqueda, lo \'unico que hace es guardar la mejor soluci\'on hasta el momento, y luego devuelve la soluci\'on que tiene o -1 si no encontr\'o soluci\'on.\\
Es trivial que el algoritmo es correcto pues recorre todo el espacio de soluciones y solamente conserva las soluciones validas y se queda con la mejor.
\subsection{Complejidad}
El algoritmo se divide en dos casos, donde se llama recursivamente con un elemento menos.\\
Definiendo la siguiente ecuaci\'on de recurrencia:\\
$$
T(n) = \left\{
\begin{array}{cl}
 \Theta (1) &\mbox{si
} n = 1 \\
2T(n)+1&\mbox{si } n > 1
\end{array}\right.
$$
\\
Como podemos ver, la recurrencia:\\ $T(n)= 2T(n-1) + 1 = 2(2T(n-1)+1)+1= ...= 2^{n-1}+ (n-1) = O(2^{n})$.\\
Con lo cual tiene complejidad: $O(2^{n})$.\\
%\newpage

%Backtracking
\section{Backtracking: Fuerza Bruta Inteligente}
\subsection{Factibilidad: recorrido siempre v\'alido}
Una camino en el \'arbol del espacio de soluciones, es v\'alido, siempre y cuando la suma de sus elementos no exceda el valor objetivo.
Con esto en mente, si encontramos un camino para el cual la suma de sus elementos es mayor al valor objetivo, al estar trabajando con enteros positivos, si seguimso agregando elementos la suma siempre exceder\'a el valor objetivo, por lo tanto cortamos esa rama del espacio de soluciones y seguimos por otra rama.\\

\subsubsection{Algoritmo}
\begin{codebox}
    \Procname{\proc{FB}(vec<int> C, int V, int i, int n) \textcolor{red}{O($2^n$)}}
    \li \If $i <= n$ \quad\quad\quad\quad\quad\quad\quad\textcolor{red}{O($1$)}
        \Then
    \li        \If $V == 0$ : \quad\quad\quad\quad\quad\quad\textcolor{red}{O($1$)}
                \Then
    \li             \Return $0$ \quad\quad\quad\quad\quad\quad\quad\textcolor{red}{O($1$)}
    \li         \Else: 
    \li                 \Return $Inf$\footnote{Si el algor\'itmo devulve un n\'umero mayor que $n$ significa que no existe soluci\'on} \quad\quad\quad\quad\quad\quad\quad\textcolor{red}{O($1$)}
                \End
                \End

    \li \Return min(FB($C$,$V$,$i+1$,$n$),$1 +$FB($C$, $V-C_{i}$, $i+1$, $n$)) \textcolor{red}{O($2^n$)}

    \end{codebox}
\subsubsection{Correctitud}
El algoritmo recorre solomente las ramas que tienen soluci\'on, con lo cual, llega a las mismas soluciones que Fuerza Bruta y lo \'unico que hace es quedarse con la mejor soluci\'on.
\subsubsection{Complejidad}
Vemos que el algoritmo define la misma recurrencia que fuerza bruta:\\
$$
T(n) = \left\{
\begin{array}{cl}
 \Theta (1) &\mbox{si
} n = 1 \\
2T(n)+1&\mbox{si } n > 1
\end{array}\right.
$$
\\
Con lo cual tiene complejidad: $O(2^{n})$.\\
\newpage
\subsection{Optimalidad: Recorrido siempre optimo}
Para nuestro problema, una soluci\'on es \'optima, si la cantidad de elementos de la soluci\'on es menor o igual que la cantidad de elementos para toda otra soluci\'on.\\
Dado una rama del espacio de b\'usqueda definido por el recorrido, si la cantidad de elementos de la rama es mayor o igual a la cantidad de elementos de la mejor soluci\'on encontrada hasta el momento, entonces se poda esa rama.\\
Dada, la poda solo corta las ramas con mayor o igual cantidad de elementos que pueden o no llegar a una soluci\'on\\
\subsubsection{Algoritmo}
\begin{codebox}
    \Procname{\proc{FB}(vec<int> C, int V, int i, int n) \textcolor{red}{O($2^n$)}}
    \li \If $i <= n$ \quad\quad\quad\quad\quad\quad\quad\textcolor{red}{O($1$)}
        \Then
    \li        \If $V == 0$ : \quad\quad\quad\quad\quad\quad\textcolor{red}{O($1$)}
                \Then
    \li             \Return $0$ \quad\quad\quad\quad\quad\quad\quad\textcolor{red}{O($1$)}
    \li         \Else: 
    \li                 \Return $Inf$\footnote{Si el algor\'itmo devulve un n\'umero mayor que $n$ significa que no existe soluci\'on} \quad\quad\quad\quad\quad\quad\quad\textcolor{red}{O($1$)}
                \End
                \End

    \li \Return min(FB($C$,$V$,$i+1$,$n$),$1 +$FB($C$, $V-C_{i}$, $i+1$, $n$)) \textcolor{red}{O($2^n$)}

    \end{codebox}
\subsubsection{Correctitud}
Supongamos que existe soluci\'on y el algoritmo no encuentra, el algoritmo no poda (hace Fuerza Bruta) hasta encontrar una soluci\'on, como no encuentra soluci\'on Fuerza Bruta no encuentra soluci\'on. \\
Absurdo pues Fuerza Bruta explora todo el espacio de soluciones.\\
Por lo tanto, si hay soluci\'on la encuentra.\\
Supongamos que hay una soluci\'on \'optima $S$ de cardinal $k$ y el algoritmo no lo encuentra:\\
Sea$ n = |S'|$ ($S'$ la soluci\'on retornada con cardinal $n$ y $n > k$), entonces antes de encontrar a $S'$, no habi\'a soluci\'on o la anterior soluci\'on tenia mayor cardinal.\\

\begin{itemize}
	\item Caso Sin Soluci\'on anterior: El algoritmo poda toda soluci\'on de cardinal $>= n$.
	\item Caso Hab\'ia Soluci\'on anterior: Sea $m$ el cardinal de la soluci\'on anterior, sabemos que $n < m$.\\
	Se podaron las soluciones $>= m$, en particular no se podo ninguna soluci\'on con cardinal $< n$.\\
\end{itemize}	
Absurdo pues $k < n$ con lo cual $S$ fue podado.
Entonces si existe una soluci\'on \'optima, el algoritmo lo encuentra y lo devuelve.
\subsubsection{Complejidad}
Vemos que algoritmo define la misma recurrencia que fuerza bruta:\\
$$
T(n) = \left\{
\begin{array}{cl}
 \Theta (1) &\mbox{si
} n = 1 \\
2T(n)+1&\mbox{si } n > 1
\end{array}\right.
$$
\\
Con lo cual tiene complejidad: $O(2^{n})$.\\
%\newpage


%Dinámica
\section{Programación Dinámica: Colision y Memoizacion}
\subsection{Principio del Optimalidad: para Subset Sum}
Definimos:

 \begin{equation}
     \label{eq:F-recursiva}
     F(S,V) = \left\{
	       \begin{array}{ll}
		 0      & \mathrm{si\ }  S = \{\} \hspace{4mm} y\hspace{4mm} V = 0\\
		 min(1+F(S-S_{n},V-valor(S_{n})),F(S-S_{n},V)) & \mathrm{si\ } n \geq 1 \\
		 \infty     & \mathrm{si\ } S = \{\} \hspace{4mm} y \hspace{4mm} V < 0
	       \end{array}
	     \right.
   \end{equation}
Sea $S1$ la solución óptima usando $X_{1},...,X_{n-1}$, tal que sumen $V-X_{n}$.\\
Sea $S2$ la solución óptima usando $X_{1},...,X_{n-1}$, tal que sumen $V$.\\
Sea $S = min(S1, S2)$, la solución óptima para el problema de $n-$elementos, entonces hay 2 opciones:\\
Usar el $n-$esimo elementos, o no usarlo.\\
Supongamos que $S$ no es solución óptima, como $S$ no es óptimo entonces existe $S'$ tal que $|S'|<|S|$.\\
\begin{itemize}
    \item Caso $X_{n}$ pertenece a $S$: $|S'| < |S| = |S1|+1$, entonces $|S'|-1 < |S1|$, como $S1$ es el óptimo para sumar $V-X_{n}$ usando $X_{1},...,X_{n-1}$ y $S'-X_{n}$ es un conjunto de elementos que suman $V-X_{n}$ y es menor que el óptimo. Absurdo pues $S1$ es óptimo.
    \item Caso $X_{n}$ no pertenece a $S$: $|S'| < |S| = |S2|$. Absurdo pues $S2$ es el óptimo para sumar V usando $X_{1},...,X_{n-1}.$ 
\end{itemize}
Con lo cuál, podemos usar Programaci\'on din\'amica para resolver el problema y F(S,V) es la funci\'on que calcula el valor optimo.

\subsection{Algoritmo}
\begin{codebox}
    \Procname{\proc{PD}(S,matrix M, int f, int c) \textcolor{green}{O($V*n$)}}
    \li \If $c = 0:$\quad\quad\textcolor{green}{O($1$)}
        \Then
    \li     \Return $0$
        \End
    \li        \If $M[f][c] == \infinity:$ \quad\quad\textcolor{green}{O($1$)}
                \Then
                \li $M[f][c] \leftarrow min(PD(S,M,f-1,c),1+PD(S,M,f-1,c-S_{f}))$
                \End

    \li \Return $M[f][c]$
    \end{codebox}
    \begin{codebox}
    \Procname{\proc{ResolverPD}(S, int V, int i, int n) \quad\quad\textcolor{green}{O($V*n$)}}
    \li $sol \leftarrow PD(S,V,i+1,n)$
    \li \If $sol = \infinity:$ \quad\quad\textcolor{green}{O($1$)}
        \Then
    \li   
            \Return $-1$
        \End
    \li     \Return $sol$

    \end{codebox}    
\subsection{Correctitud}
La demostraci\'on de correctitud es directa porque algoritmo calcula exactamente la función, con lo cual el algoritmo es correcto.\\
\subsection{Complejidad}
La complejidad es la cantidad de llamadas recursivas, como se puede ver, la función decrece en dos direcciones, llenando la matriz. Y la cantidad de llamados que se hace en peor caso es $V*n$ por que cada subproblema se resuelve una ves y se guarda en la matriz para ser accedido si se vuelve a solicitar.\\
Por lo tanto la complejidad es $O(V*n)$


%Experimentación
\section{Experimentación}
\subsection{Hip\'otesis}
	Con los experimentos se busca comprobar las siguientes hipotesis:\\
	1. Hacer un sort de los elementos de mayor a menor, hace que las podas sean m\'as efectivas.\\
	2. Hacer un sort de los elementos de menor a mayor, hace que las podas empeoren.\\
	3. La podas podas empeora a medida que el valor de los elementos se acercan a 0, y mejora a medida que el valor de los elementos se acercan al valor objetivo.\\
	4. Fuerza Bruta depende únicamente de la cantidad de items.\\
	5. Las podas de factibilidad y optimalidad en mejor caso tienen complejidad lineal.\\
\subsection{Consideraciones para las experimentaciones}
\subsubsection{Generaci\'on de elementos}
\begin{itemize}
	\item Se usa la función sort de c++ (ordena los elementos en forma creciente)
	\item Los conjuntos fueron generados aleatoriamente, con la funci\'on rand de c++ (distribución uniforme) para poder analizar en caso promedio que pasaba con cada algoritmo, en el rango entre [0..$V$].
	\item la cantidad de iteraciones es igual a 50, pues se logro ver durante la experimentaci\'on que los algoritmos se estabilizaban.
	\item El rango para la suma fue [15000...800000], porque es limitada por Programación Dinámica, la cual pide memoria de tamaño $n*W$ y al ser $W$ tan grande, hace que no se pueda seguir experimentando.
	\item El rango para la cantidad de elementos fue [5...35], porque al querer analizar el caso promedio se toma el promedio de 50 iteraciones y los algoritmos exponenciales tardan demasiado.
\end{itemize}
\subsubsection{Im\'agenes}
\begin{itemize}
	\item La etiqueta $decreciente$ indica que los elementos fueron ordenados de menor a mayor para hacer la experimentaci\'on.
	\item La etiqueta $creciente$ indica que los elementos fueron ordenados de menor a mayor para hacer la experimentaci\'on.
	\item La etiqueta $unSort$ indica que los elementos no fueron ordenados.
\end{itemize}	
\subsection{Complejidades Te\'oricas en la pr\'actica}
\subsubsection{Fuerza Bruta}
La Figura 1 muestra como Fuerza Bruta no es afectada por el sort de los elemento, y al aplicarle el logaritmo en base dos se logra ver que es lineal, lo cual es esperado pues es exponencial en la cantidad de elementos con base igual a 2.
\begin{figure}[h]
\centering
	\includegraphics[width=0.6\textwidth]{fbSort.png}
\caption{Relación entre el orden de los elementos y la complejidad temporal}
\end{figure}

\subsubsection{Backtracking poda por factibilidad}
Tener los elementos ordenados de mayor a menor hace que la poda sea mas efectiva pues al considerar los elementos de mayor valor primero el algoritmo se ahorra comparaciones y empeora cuando el ordenamiento es de menor a mayor.\\ 
Por ejemplo: $Conj1 = \{1,2\}$, $Conj2 = \{2,1\}$ y el valor objetivo 2. \\
Para el $Conj1$, considera los subconjuntos $\{ 1\}$, $\{ 1,2\}$, y $\{2\}$.\\
Para el $Conj2$, considera los subconjuntos $\{2\}$ pues no necesita agregar nada mas para llegar al valor objetivo y $\{1\}$.\\
El algoritmo hace una comparación menos, pero si el conjunto es mas grande podría hacer menos comparaciones si la suma de los elementos exceden el valor objetivo.\\
Se aplica logaritmo en base dos y se logra ver que el resultado es lineal.\\
Los picos de la función se deben a las podas realizadas.\\
\begin{figure}[h]
\centering
\includegraphics[width=0.6\textwidth]{facSort.png}
\caption{Relación entre el orden de los elementos y la complejidad temporal}
\end{figure}

\subsubsection{Backtracking poda por optimalidad}
El ordenamiento de los elementos afecta a la poda de optimalidad.\\
Por ejemplo: $Conj1 = \{3,4\}$, $Conj2 = \{4,3\}$ y el valor objetivo 4. \\
Para el $Conj1$, considera los subconjuntos $\{ 3\}$, $\{ 3,4\}$, y $\{4\}$.\\
Para el $Conj2$, considera los subconjuntos $\{4\}$ por que la suma del primer subconjunto es el valor objetivo y el cardinal óptimo es 1, entonces no considera conjuntos mayores o iguales a 1\\
\begin{figure}[h]
\centering
\includegraphics[width=0.6\textwidth]{opSort.png}
\caption{Relación entre el orden de los elementos y la complejidad temporal}
\end{figure}

\subsubsection{Programación dinámica}
En este experimento se fija $n=35$ y se varia $V$ para ver como afecta a la complejidad temporal.\\
En la figura 4 se ve como Programacion Dinamica es afectado directamente por el valor objetivo y en menor medida por el ordenamiento de los elementos.
\begin{figure}[h]
\centering
\includegraphics[width=0.6\textwidth]{pdObjetivo.png}
\caption{Relación entre el orden de los elementos, el valor objetivo y la complejidad temporal}
\end{figure}

\subsubsection{Influencia del valor de los elementos}
Para este experimento se fija $V=800000$ y $n=25$.\\
En la figura 5 el valor de los elementos se genera entre $[0,...,V/i]$ para la posición i.\\
Se logra ver que a medida que los elementos decrecen las podas no son tan efectivas.\\
Fuerza Bruta no es afectado por el valor de los elementos.\\
Programaci\'on dinámica es menos afectada por el decrecimiento del valor de los elementos.\\
\begin{figure}[h]
\centering
\includegraphics[width=0.6\textwidth]{todostodos.png}
\caption{Relación entre el decrecimiento de el valor de los elementos y como afecta a las complejidades}
\end{figure}\\
Este caso fue generado para ver el comportamiento lineal de estos algoritmos en mejor caso.\\
Los elementos son exactamente el valor objetivo y a medida que va creciendo la cantidad de elementos la complejidad temporal dividida por la cantidad de elementos se ajusta mejor.
\begin{figure}[h]
\centering
\includegraphics[width=0.6\textwidth]{mejorcasoBack.png}
\caption{Mejor caso para las podas, tiempo lineal}
\end{figure}
%\newpage

%Conclusión
\section{Conclusión}
Los algoritmos de Backtracking no siempre siempre mejores que usar fuerza bruta.\\
Programaci\'on din\'amica es bastante buena para valores chicos, pero logramos ver como crece r\'apidamente al incrementar el valor de la suma.\\
Como Fuerza Bruta y los algoritmos de Backtraking no son afectados por el valor de la suma, para valores muy grandes estos algoritmos podr\'ian tener un mejor rendimiento.\\
Para resolver este tipo de problema, primero hay que hacer un analisis del entorno donde se va a usar, para saber si hay una cota o no, luego de eso se elige el algoritmo para ser usado. Si hay cota y es chica en comparaci\'on con la suma, se usará PD, caso contrario alguno de los algoritmos exponenciales los cuales no resultaros ser muy diferentes.

%\newpage

%changelog
\section{changelog}
Se agregaron los pseudocogidos correspondientes a cada algoritmo.\\
Se juntaron las podas en una sección.\\
Se paso del algoritmo bottom-up al algoritmo top-down.\\
Justificación de principio de optimalidad por Absurdo.\\
La demostración de correctitud quedó directa despues de definir la función recursiva que calcula la solución.\\
Se agregó experimentación para los 4 algoritmos.\\


%\newpage

\end{document}
