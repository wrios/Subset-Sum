\subsection{Principio del Optimalidad: para Subset Sum}
Definimos:

 \begin{equation}
     \label{eq:F-recursiva}
     F(S,V) = \left\{
	       \begin{array}{ll}
		 0      & \mathrm{si\ }  S = \{\} \hspace{4mm} y\hspace{4mm} V = 0\\
		 min(1+F(S-S_{n},V-valor(S_{n})),F(S-S_{n},V)) & \mathrm{si\ } n \geq 1 \\
		 \infty     & \mathrm{si\ } S = \{\} \hspace{4mm} y \hspace{4mm} V < 0
	       \end{array}
	     \right.
   \end{equation}
Sea $S1$ la solución óptima usando $X_{1},...,X_{n-1}$, tal que sumen $V-X_{n}$.\\
Sea $S2$ la solución óptima usando $X_{1},...,X_{n-1}$, tal que sumen $V$.\\
Sea $S = min(S1, S2)$, la solución óptima para el problema de $n-$elementos, entonces hay 2 opciones:\\
Usar el $n-$esimo elementos, o no usarlo.\\
Supongamos que $S$ no es solución óptima, como $S$ no es óptimo entonces existe $S'$ tal que $|S'|<|S|$.\\
\begin{itemize}
    \item Caso $X_{n}$ pertenece a $S$: $|S'| < |S| = |S1|+1$, entonces $|S'|-1 < |S1|$, como $S1$ es el óptimo para sumar $V-X_{n}$ usando $X_{1},...,X_{n-1}$ y $S'-X_{n}$ es un conjunto de elementos que suman $V-X_{n}$ y es menor que el óptimo. Absurdo pues $S1$ es óptimo.
    \item Caso $X_{n}$ no pertenece a $S$: $|S'| < |S| = |S2|$. Absurdo pues $S2$ es el óptimo para sumar V usando $X_{1},...,X_{n-1}.$ 
\end{itemize}
Con lo cuál, podemos usar Programaci\'on din\'amica para resolver el problema y F(S,V) es la funci\'on que calcula el valor optimo.

\subsection{Algoritmo}
\begin{codebox}
    \Procname{\proc{PD}(S,matrix M, int f, int c) \textcolor{green}{O($V*n$)}}
    \li \If $c = 0:$\quad\quad\textcolor{green}{O($1$)}
        \Then
    \li     \Return $0$
        \End
    \li        \If $M[f][c] == \infinity:$ \quad\quad\textcolor{green}{O($1$)}
                \Then
                \li $M[f][c] \leftarrow min(PD(S,M,f-1,c),1+PD(S,M,f-1,c-S_{f}))$
                \End

    \li \Return $M[f][c]$
    \end{codebox}
    \begin{codebox}
    \Procname{\proc{ResolverPD}(S, int V, int i, int n) \quad\quad\textcolor{green}{O($V*n$)}}
    \li $sol \leftarrow PD(S,V,i+1,n)$
    \li \If $sol = \infinity:$ \quad\quad\textcolor{green}{O($1$)}
        \Then
    \li   
            \Return $-1$
        \End
    \li     \Return $sol$

    \end{codebox}    
\subsection{Correctitud}
La demostraci\'on de correctitud es directa porque algoritmo calcula exactamente la función, con lo cual el algoritmo es correcto.\\
\subsection{Complejidad}
La complejidad es la cantidad de llamadas recursivas, como se puede ver, la función decrece en dos direcciones, llenando la matriz. Y la cantidad de llamados que se hace en peor caso es $V*n$ por que cada subproblema se resuelve una ves y se guarda en la matriz para ser accedido si se vuelve a solicitar.\\
Por lo tanto la complejidad es $O(V*n)$