\subsection{principio del Optimalidad: para Subset Sum}
Sea $S1$ la solucion optima usando $X_{1},...,X_{n-1}$, tal que sumen $V-X_{n}$\\
Sea $S2$ la solucion optima usando $X_{1},...,X_{n-1}$, tal que sumen $V$\\
Sea S = min(S1, S2), la solucion optima para el problema de n-elementos, entonces hay 2 opciones:\\
Usar el n-esimo elementos, o no usarlo.\\
Supongamos que $S$ no es solucion optima, como S no es optimo entonces existe S' tal que |S'|<|S|.\\
\begin{itemize}
    \item Caso $X_{n}$ pertenece a $S$: $|S'| < |S| = |S1|+1$, entonces $|S'|-1 < |S1|$, como $S1$ es el optimo para sumar $V-X_{n}$ usando $X_{1},...,X_{n-1}$ y $|S'-X_{n}|$ es un conjunto de elementos que suman $V-X_{n}$ y es menor que el optimo. Absurdo pues $S1$ es optimo.
    \item Caso $X_{n}$ no pertenece a $S$: $|S'| < |S| = |S2|$. Absurdo pues $S2$ es el optimo para sumar V usando $X_{1},...,X_{n-1}$ 
\end{itemize}
Vimos en la introducci\'on que las soluciones de este problema pueden ser expresados mediante una secuencia de decisiones.\\
Ahora para que ver cumple el principio de optimalidad definimos:

 \begin{equation}
     \label{eq:F-recursiva}
     F(x) = \left\{
	       \begin{array}{ll}
		 0      & \mathrm{si\ }  C = \{\} \hspace{4mm} y\hspace{4mm} S = 0\\
		 min(1+F(C-V_{n},S-valor(V_{n})),F(C-V_{n},S)) & \mathrm{si\ } n > 1 \\
		 \infty     & \mathrm{si\ } C = \{\} \hspace{4mm} y \hspace{4mm} S < 0
	       \end{array}
	     \right.
   \end{equation}
Llamamos al problema \'optimo de resolver el problema original con un elemento menos(usando el n-esimo elemento y sin usar el n-esimo elemento que son los dos \'unicos casos), osea dos subproblemas \'optimos. Por lo tanto con esos dos subproblemas se puede armar el problema mas grande.\\
Con lo cual, podemos usar Programaci\'on din\'amica para resolver el problema.

\subsection{Algoritmo}
\begin{codebox}
    \Procname{\proc{FB}(vec$\<$int$\>$ C, int V, int i, int n) \textcolor{red}{O($2^n$)}}
    \li \If $i <= n$ \quad\quad\quad\quad\quad\quad\quad\textcolor{red}{O($1$)}
        \Then
    \li        \If $V == 0$ : \quad\quad\quad\quad\quad\quad\textcolor{red}{O($1$)}
                \Then
    \li             \Return $0$ \quad\quad\quad\quad\quad\quad\quad\textcolor{red}{O($1$)}
    \li         \Else: 
    \li                 \Return $Inf$\footnote{Si el algor\'itmo devulve un n\'umero mayor que $n$ significa que no existe soluci\'on} \quad\quad\quad\quad\quad\quad\quad\textcolor{red}{O($1$)}
                \End
                \End

    \li \Return min(FB($C$,$V$,$i+1$,$n$),$1 +$FB($C$, $V-C_{i}$, $i+1$, $n$)) \textcolor{red}{O($2^n$)}

    \end{codebox}
\subsection{Correctitud}
Dado que subset-sum cumple el principio de optimalidad, se puede usar esa forma recursiva para poder calcular el valor optimo.\\
Ahora veremos que el algoritmo devuelve lo mismo que la funcion.\\
\subsection{Complejidad}
La complejidad es f\'acil de deducir, se recorren dos array de cardinal V y se llama a la iteraci\'on n veces.\\
Por lo tanto la complejidad es O(V*n)